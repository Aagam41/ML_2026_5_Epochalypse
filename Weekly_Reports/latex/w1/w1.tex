\documentclass[11pt,a4paper]{article}
\usepackage[margin=1in]{geometry}
\usepackage{graphicx}
\usepackage{hyperref}
\usepackage{enumitem}
\usepackage{titlesec}
\usepackage{fancyhdr}

% Header and footer
\pagestyle{fancy}
\fancyhf{}
\rhead{Week 1 Report}
\lhead{MOT using Classical ML}
\rfoot{Page \thepage}

\title{\textbf{Weekly Progress Report - Week 1} \\
	\large Multiple Object Tracking using Classical Machine Learning \\
	for Re-Identification in UAV Videos}

\author{Group: Epochalypse\\\{Aagam Sheth, Mahima Parekh, Aakanksha Jadhav, Vansh Lilani\}}

\date{Reporting Period: February 9 - February 14, 2026 \\
	Submission Date: February 14, 2026}


\begin{document}

\maketitle

\section*{Outline of Performed Tasks:}

\begin{itemize}[leftmargin=*]
    \item Reviewed reference materials provided by TA including CVPR 2025 paper, MOT metrics documentation, and GitHub repositories for Kalman Filter implementation, and BoxMOT framework
    \item Successfully downloaded and explored the AU Drone Dataset
    \item Studied classical tracking algorithms: SORT, Mean-Shift, CamShift, and Kalman Filtering
\end{itemize}

\section*{Project Overview}

\textbf{Problem Statement:} Improve existing online trackers by measuring feature evolution of objects using Classical Machine Learning models for Re-Identification in UAV videos. This study evaluates Traditional Machine Learning for ReID in object tracking, moving away from deep learning. Instead of neural networks, it uses hand-crafted features like shape and color to create an object's identity.

\section*{Reference Materials Studied}

\begin{enumerate}[leftmargin=*]
    \item Primary research paper from CVPR 2025: \\
    \url{https://cvpr.thecvf.com/virtual/2025/poster/35174}
    \item MOT tracking metrics documentation: \\
    \url{https://miguel-mendez-ai.com/2024/08/25/mot-tracking-metrics}
    \item GitHub repositories for implementation reference:
    \begin{itemize}
        \item Multiple Object Tracking using Kalman Filter: \\
        \url{https://github.com/NickNair/Multiple-Object-Tracking-using-Kalman-Filter}
        \item BoxMOT tracker framework: \\
        \url{https://github.com/mikel-brostrom/boxmot}
    \end{itemize}
\end{enumerate}

\section*{Classical Tracking Algorithms Studied}

\textbf{SORT (Simple Online and Realtime Tracking):} Studied core architecture combining Kalman Filter for motion prediction and Hungarian algorithm for data association. Analyzed limitations related to appearance-based ReID.

\textbf{Mean-Shift and CamShift Trackers:} Examined color histogram-based tracking mechanisms and adaptive window sizing in CamShift for varying object scales in UAV scenarios.

\textbf{Kalman Filtering:} Reviewed mathematical foundations, prediction and update equations for motion modeling in object tracking context.

\section*{Outcomes:}

\begin{itemize}[leftmargin=*]
    \item Understood classical tracking methods (SORT, Mean-Shift, CamShift) and Kalman Filtering
    \item Understood AU Drone Dataset structure
\end{itemize}

\section*{Tentative List of Tasks for Next Week:}

\begin{enumerate}[leftmargin=*]
    \item Set up development environment (Python, OpenCV, PyTorch/TensorFlow, YOLO v8) and complete study of reference papers
    \item Study feature extraction module (HOG, Color Histograms) and test on sample videos
    \item Understand MOT evaluation metrics, (MOTA, MOTP, IDF1)
    \item Start with dataset preprocessing pipeline for AU Drone Dataset
\end{enumerate}

\end{document}
